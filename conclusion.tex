\begin{table*}[ht!]
    \caption{Predictions for the progress in speech recognition research and applications
    by the year 2030.}
    \centering
    \begin{tabular}{l}
    \toprule
    Prediction \\
    \midrule
    Self-supervised learning and pretrained models are here to stay. \\
    \rowcolor{Gray} Most speech recognition (inference) will happen at the edge. \\
    On-device model training will be much more common. \\
    \rowcolor{Gray} Sparsity will be a key research direction to enable on-device inference and training. \\
    Improving word error rate on common benchmarks will fizzle out as a research goal. \\
    \rowcolor{Gray} Speech recognizers will output richer representations (graphs) for use by downstream tasks. \\
    Personalized models will be commonplace. \\
    \rowcolor{Gray} Most transcription services will be automated. \\
    Voice assistants will continue to improve, but incrementally. \\
    \rowcolor{Gray} We will not live in always listening, voice-controlled homes. \\
    Voice controlled augmented reality glasses will not be widespread. \\
    \bottomrule
    \end{tabular}
    \label{tab:predictions}
\end{table*}

\section{Conclusion}
\label{sec:conclusion}

Table~\ref{tab:predictions} summarizes my predictions for the progress in
speech recognition to the year 2030. The predictions show that the coming
decade could be just as exciting and important to the development of speech
recognition and spoken language understanding as the previous one. We still
have many research problems to solve before speech recognition has reached the
point where it works all the time, for everyone. However, this is a goal worth
working toward, as speech recognition is a key component to more fluid,
natural, and accessible interactions with technology.
