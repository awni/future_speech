\begin{table*}[ht!]
    \caption{Predictions for the progress in speech recognition research and applications
    by the year 2030.}
    \centering
    \begin{tabular}{l}
    \toprule
    Prediction \\
    \midrule
    Self-supervised learning and pretrained models are here to stay. \\
    \rowcolor{Gray} Most speech recognition (inference) will happen at the edge. \\
    On-device model training will be much more common. \\
    \rowcolor{Gray} Sparsity will be a key research direction to enable on-device inference and training. \\
    Improving word error rate on common benchmarks will fizzle out as a research goal. \\
    \rowcolor{Gray} Speech recognizers will output richer representations (graphs) for use by downstream tasks. \\
    Personalized models will be commonplace. \\
    \rowcolor{Gray} Most transcription services will be automated. \\
    Voice assistants will continue to improve, but incrementally. \\
    \bottomrule
    \end{tabular}
    \label{tab:predictions}
\end{table*}

\section{Application Predictions}
\label{sec:application_predictions}

\subsection{Transcription Services}

{\bf Prediction:} By the end of the decade, 99\% of transcribed speech services
will be done by automatic speech recognition. Human transcribers will perform
quality control and correct or transcribe the more difficult utterances.
Transcription services include, for example, captioning video, transcribing
interviews, and transcribing lectures or speeches.

\subsection{Voice Assistants}

{\bf Prediction:} Voice assistants will get better, but incrementally, not
fundamentally. Speech recognition is no longer the bottleneck to better voice
assistants. The bottlenecks are now fully in the language understanding domain
including the ability to maintain conversations, multi-ply contextual
responses, and much wider domain question and answering. We will continue to
make incremental progress on these so-called AI-complete
problems,\footnote{\citet[sec. 4]{shapiro1992encyclopedia} defines an AI task as
AI-complete if solving it is equivalent to ``solving the general AI problem'',
which he defines as ``producing a generally intelligent computer program''.}
but I don't expect them to be solved by 2030.

Will we live in smart homes that are always listening and can respond to our
every vocal beck and call? No. Will we wear augmented reality glasses on our
faces and control them with our voice? Not by 2030.

